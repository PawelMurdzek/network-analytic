\documentclass[12pt,a4paper]{article}

% Kodowanie i język
\usepackage[utf8]{inputenc}
\usepackage[T1]{fontenc}
\usepackage[polish]{babel}

% Marginesy
\usepackage[margin=2.5cm]{geometry}

% Grafika i tabele
\usepackage{graphicx}
\usepackage{float}
\usepackage{booktabs}
\usepackage{longtable}
\usepackage{array}
\usepackage{multirow}
\usepackage{colortbl}
\usepackage{xcolor}

% Matematyka
\usepackage{amsmath}
\usepackage{amsfonts}

% Linki i odnośniki
\usepackage{hyperref}
\hypersetup{
    colorlinks=true,
    linkcolor=blue,
    urlcolor=blue,
    citecolor=blue
}

% Nagłówki i stopki
\usepackage{fancyhdr}
\pagestyle{fancy}
\fancyhf{}
\rhead{Analiza Ruchu Sieciowego}
\lhead{\leftmark}
\rfoot{Strona \thepage}

% Listingi kodu
\usepackage{listings}
\lstset{
    basicstyle=\ttfamily\small,
    breaklines=true,
    frame=single,
    backgroundcolor=\color{gray!10}
}

% Definicje kolorów
\definecolor{headerblue}{RGB}{102,126,234}
\definecolor{alertred}{RGB}{244,67,54}
\definecolor{alertorange}{RGB}{255,152,0}
\definecolor{alertgreen}{RGB}{76,175,80}

% Niestandardowe komendy
\newcommand{\alerthigh}[1]{\colorbox{alertred!20}{\textcolor{alertred}{\textbf{#1}}}}
\newcommand{\alertmedium}[1]{\colorbox{alertorange!20}{\textcolor{alertorange}{\textbf{#1}}}}
\newcommand{\alertlow}[1]{\colorbox{alertgreen!20}{\textcolor{alertgreen}{\textbf{#1}}}}


\begin{document}

\begin{titlepage}
    \centering
    \vspace*{2cm}
    
    \rule{\textwidth}{1.5pt}
    \vspace{0.5cm}
    
    {\Huge \textbf{Raport Analizy Ruchu Sieciowego}}
    
    \vspace{0.5cm}
    \rule{\textwidth}{1.5pt}
    
    \vspace{2cm}
    
    {\Large System Detection as a Code}\\[0.5cm]
    {\Large + Machine Learning + Threat Intelligence}
    
    \vspace{3cm}
    
    {\large \textbf{Autor:} System Analizy Sieciowej}\\[0.5cm]
    {\large \textbf{Data:} 30.11.2025, 14:33}
    
    \vspace{2cm}
    
    \includegraphics[width=0.3\textwidth]{example-image}
    
    \vfill
    
    {\small Wygenerowano automatycznie przez system analizy sieciowej}
    
\end{titlepage}

\tableofcontents
\newpage

\section{Wprowadzenie}

Niniejszy raport przedstawia wyniki analizy ruchu sieciowego przeprowadzonej za pomocą zaawansowanego systemu detekcji zagrożeń. System wykorzystuje następujące komponenty:

\begin{itemize}
    \item \textbf{Analiza przepływów (Flow Analysis)} -- przetwarzanie plików PCAP przy użyciu NFStream/Scapy
    \item \textbf{Detection as a Code} -- reguły detekcyjne w Pythonie oraz format Sigma
    \item \textbf{Machine Learning} -- klasyfikacja przepływów przy użyciu algorytmów ML
    \item \textbf{Threat Intelligence} -- wzbogacanie danych o informacje geolokalizacyjne i reputacyjne
\end{itemize}

\subsection{Metodologia}

Analiza została przeprowadzona zgodnie z następującymi krokami:
\begin{enumerate}
    \item Wczytanie i parsowanie pliku PCAP
    \item Ekstrakcja przepływów sieciowych (5-tuple)
    \item Wykonanie reguł detekcyjnych
    \item Klasyfikacja ML
    \item Wzbogacenie o dane Threat Intelligence
    \item Generowanie wizualizacji i raportu
\end{enumerate}


\section{Statystyki Przepływów}
\label{sec:statistics}

Ta sekcja przedstawia podstawowe statystyki analizowanego ruchu sieciowego.

\subsection{Podsumowanie}

\begin{table}[H]
\centering
\caption{Podstawowe statystyki przepływów}
\begin{tabular}{|l|r|}
\hline
\rowcolor{headerblue!20}
\textbf{Metryka} & \textbf{Wartość} \\
\hline
Całkowita liczba przepływów & 390 \\
Unikalne IP źródłowe & 108 \\
Unikalne IP docelowe & 40 \\
Całkowita liczba pakietów & 430 \\
Całkowita liczba bajtów & 70,332 \\
Średnia pakietów/przepływ & 1.10 \\
Średnia bajtów/przepływ & 180.34 \\
\hline
\end{tabular}
\end{table}

\subsection{Rozkład protokołów}

W analizowanym ruchu wykryto następujące protokoły:
\begin{itemize}
    \item TCP: 270 przepływów
    \item UDP: 119 przepływów
    \item ICMP: 1 przepływów
\end{itemize}

\subsection{Top 5 komunikacji między hostami}

\begin{table}[H]
\centering
\caption{Najczęściej komunikujące się pary hostów}
\begin{tabular}{|c|l|l|r|r|}
\hline
\rowcolor{headerblue!20}
\textbf{\#} & \textbf{IP Źródłowe} & \textbf{IP Docelowe} & \textbf{Pakiety} & \textbf{Bajty} \\
\hline
        1 & 192.168.1.88 & 8.8.8.8 & 81 & 9,430 \\
        2 & 192.168.1.200 & 192.168.1.1 & 40 & 3,360 \\
        3 & 192.168.1.99 & 192.168.1.20 & 30 & 1,200 \\
        4 & 192.168.1.120 & 198.51.100.50 & 30 & 3,812 \\
        5 & 192.168.1.50 & 203.0.113.66 & 20 & 24,751 \\
\hline
\end{tabular}
\end{table}


\section{Wykryte Zagrozenia}
\label{sec:alerts}

System detekcji zidentyfikowal \textbf{11} alertow bezpieczenstwa.

\subsection{Rozklad wedlug poziomu zagrozenia}

\begin{table}[H]
\centering
\caption{Podział alertów według severity}
\begin{tabular}{|l|c|c|}
\hline
\rowcolor{headerblue!20}
\textbf{Poziom} & \textbf{Liczba} & \textbf{Procent} \\
\hline
\alerthigh{HIGH/CRITICAL} & 1 & 9.1\% \\
\alertmedium{MEDIUM} & 10 & 90.9\% \\
\alertlow{LOW} & 0 & 0.0\% \\
\hline
\textbf{RAZEM} & \textbf{11} & \textbf{100\%} \\
\hline
\end{tabular}
\end{table}

\subsection{Lista wykrytych alertów}

\begin{longtable}{|c|p{4cm}|c|p{6cm}|}
\hline
\rowcolor{headerblue!20}
\textbf{\#} & \textbf{Reguła} & \textbf{Poziom} & \textbf{Wiadomość} \\
\hline
\endfirsthead
\hline
\rowcolor{headerblue!20}
\textbf{\#} & \textbf{Regula} & \textbf{Poziom} & \textbf{Wiadomosc} \\
\hline
\endhead
        1 & Port Scan Detection & \alerthigh{HIGH} & Możliwe skanowanie portów z 192.168.1.99: 30 różny \\
        2 & Suspicious Port Connection & \alertmedium{MEDIUM} & Połączenie do podejrzanego portu: 192.168.1.75 ->  \\
        3 & Suspicious Port Connection & \alertmedium{MEDIUM} & Połączenie do podejrzanego portu: 192.168.1.75 ->  \\
        4 & Suspicious Port Connection & \alertmedium{MEDIUM} & Połączenie do podejrzanego portu: 192.168.1.75 ->  \\
        5 & Suspicious Port Connection & \alertmedium{MEDIUM} & Połączenie do podejrzanego portu: 192.168.1.75 ->  \\
        6 & Suspicious Port Connection & \alertmedium{MEDIUM} & Połączenie do podejrzanego portu: 192.168.1.75 ->  \\
        7 & Suspicious Port Connection & \alertmedium{MEDIUM} & Połączenie do podejrzanego portu: 192.168.1.75 ->  \\
        8 & Suspicious Port Connection & \alertmedium{MEDIUM} & Połączenie do podejrzanego portu: 192.168.1.75 ->  \\
        9 & Suspicious Port Connection & \alertmedium{MEDIUM} & Połączenie do podejrzanego portu: 192.168.1.75 ->  \\
        10 & Suspicious Port Connection & \alertmedium{MEDIUM} & Połączenie do podejrzanego portu: 192.168.1.75 ->  \\
        11 & Suspicious Port Connection & \alertmedium{MEDIUM} & Połączenie do podejrzanego portu: 192.168.1.75 ->  \\
\hline
\end{longtable}


\subsection{Dostepne reguly Sigma}

System wykorzystuje reguly Sigma zgodne z frameworkiem MITRE ATT\&CK:

\begin{table}[H]
\centering
\caption{Lista regul Sigma}
\begin{tabular}{|p{6cm}|c|c|}
\hline
\rowcolor{headerblue!20}
\textbf{Regula} & \textbf{Poziom} & \textbf{MITRE} \\
\hline
        Cobalt Strike C2 Beacon Detection & CRITICAL & 001, T1573 \\
        Cryptocurrency Mining Communication & HIGH & T1496 \\
        Large Data Transfer - Potential Exf & LOW & T1048, T1041 \\
        DNS Tunneling Detection & MEDIUM & 004, T1048 \\
        Metasploit Reverse Shell Connection & CRITICAL & T1059 \\
        Network Port Scan Detection & MEDIUM & T1046 \\
        RDP Brute Force Attempt Detection & HIGH & T1110, 001 \\
        SMB Lateral Movement Detection & HIGH & 002, T1570 \\
        SSH Connection - Potential Brute Fo & MEDIUM & 001, 004 \\
        Tor Network Traffic Detection & MEDIUM & 003 \\
\hline
\end{tabular}
\end{table}



\section{Wyniki Klasyfikacji Machine Learning}
\label{sec:ml}

Model Machine Learning został wykorzystany do klasyfikacji przepływów sieciowych jako normalne lub podejrzane.

\subsection{Metryki jakości modelu}

\begin{table}[H]
\centering
\caption{Metryki wydajności klasyfikatora}
\begin{tabular}{|l|c|l|}
\hline
\rowcolor{headerblue!20}
\textbf{Metryka} & \textbf{Wartość} & \textbf{Opis} \\
\hline
Accuracy & 1.0000 & Ogólna dokładność \\
Precision & 1.0000 & Precyzja (PPV) \\
Recall (TPR) & 1.0000 & True Positive Rate \\
F1 Score & 1.0000 & Średnia harmoniczna P i R \\
\hline
\rowcolor{alertred!10}
FPR & 0.0000 & False Positive Rate \\
TNR & 1.0000 & True Negative Rate \\
\hline
\end{tabular}
\end{table}

\subsection{Macierz konfuzji}

\begin{table}[H]
\centering
\caption{Macierz konfuzji}
\begin{tabular}{|c|c|c|}
\hline
\rowcolor{headerblue!20}
 & \textbf{Pred: Normal} & \textbf{Pred: Suspicious} \\
\hline
\textbf{Real: Normal} & 114 (TN) & 0 (FP) \\
\textbf{Real: Suspicious} & 0 (FN) & 3 (TP) \\
\hline
\end{tabular}
\end{table}

\subsection{Interpretacja}

\begin{itemize}
    \item \textbf{True Positives (TP)}: 3 -- poprawnie wykryte zagrożenia
    \item \textbf{True Negatives (TN)}: 114 -- poprawnie sklasyfikowany normalny ruch
    \item \textbf{False Positives (FP)}: 0 -- fałszywe alarmy (normalny ruch oznaczony jako zagrożenie)
    \item \textbf{False Negatives (FN)}: 0 -- pominięte zagrożenia
\end{itemize}


\section{Threat Intelligence}
\label{sec:threatintel}

Przepływy zostały wzbogacone o dane geolokalizacyjne i informacje o reputacji adresów IP.

\subsection{Top 10 krajów docelowych}

\begin{table}[H]
\centering
\caption{Rozkład geograficzny ruchu docelowego}
\begin{tabular}{|l|r|}
\hline
\rowcolor{headerblue!20}
\textbf{Kraj} & \textbf{Liczba przepływów} \\
\hline
        United States & 181 \\
        Romania & 55 \\
        Germany & 24 \\
        Canada & 21 \\
        Australia & 21 \\
        Hong Kong & 13 \\
\hline
\end{tabular}
\end{table}

\subsection{Źródła danych Threat Intelligence}

System wykorzystuje następujące źródła danych:
\begin{itemize}
    \item \textbf{ip-api.com} -- geolokalizacja IP (kraj, miasto, ISP)
    \item \textbf{AbuseIPDB} -- reputacja IP, raporty o nadużyciach (opcjonalnie)
    \item \textbf{VirusTotal} -- analiza złośliwości IP (opcjonalnie)
\end{itemize}


\section{Wizualizacje}
\label{sec:visualizations}

Poniżej przedstawiono wizualizacje wyników analizy.


\begin{figure}[H]
    \centering
    \includegraphics[width=0.9\textwidth]{protocol_distribution.png}
    \caption{Protocols}
\end{figure}


\begin{figure}[H]
    \centering
    \includegraphics[width=0.9\textwidth]{severity_distribution.png}
    \caption{Severity}
\end{figure}


\begin{figure}[H]
    \centering
    \includegraphics[width=0.9\textwidth]{alerts_timeline.png}
    \caption{Alerts}
\end{figure}


\begin{figure}[H]
    \centering
    \includegraphics[width=0.9\textwidth]{confusion_matrix.png}
    \caption{Confusion}
\end{figure}


\begin{figure}[H]
    \centering
    \includegraphics[width=0.9\textwidth]{top_dst_ip.png}
    \caption{Top Ips}
\end{figure}




\section{Podsumowanie i wnioski}
\label{sec:summary}

\subsection{Kluczowe ustalenia}

\begin{enumerate}
    \item Przeanalizowano \textbf{390} przeplywow sieciowych
    \item Wykryto \textbf{11} alertow bezpieczenstwa, w tym \textbf{1} o wysokim priorytecie
    \item Ruch sieciowy obejmowal \textbf{108} unikalnych zrodlowych i \textbf{40} docelowych adresow IP
    \item Model ML osiagnal dokladnosc \textbf{100.00%}
\end{enumerate}

\subsection{Rekomendacje}

Na podstawie przeprowadzonej analizy zaleca się:
\begin{itemize}
    \item Zbadanie alertów o wysokim priorytecie
    \item Weryfikacja podejrzanych adresów IP w systemach SIEM
    \item Aktualizacja reguł detekcyjnych na podstawie wykrytych wzorców
    \item Rozważenie blokady najbardziej podejrzanych adresów IP na firewallu
\end{itemize}

\subsection{Spełnione wymagania projektowe}

\begin{table}[H]
\centering
\caption{Lista spełnionych wymagań}
\begin{tabular}{|c|l|c|}
\hline
\rowcolor{headerblue!20}
\textbf{ID} & \textbf{Wymaganie} & \textbf{Status} \\
\hline
A.1 & Wczytywanie PCAP (NFStream/Scapy) & \checkmark \\
A.2 & Statystyki przepływów & \checkmark \\
D.1 & Reguły detekcyjne Python & \checkmark \\
D.2 & Reguły Sigma & \checkmark \\
ML.1 & Klasyfikacja ML & \checkmark \\
ML.2 & Metryki FPR/TPR & \checkmark \\
ML.3 & Trenowanie na nowych danych & \checkmark \\
E.1 & Threat Intelligence enrichment & \checkmark \\
V.1 & Wizualizacje alertów & \checkmark \\
V.2 & Mapa geograficzna & \checkmark \\
\hline
\end{tabular}
\end{table}


\end{document}
